
% Default to the notebook output style

    


% Inherit from the specified cell style.




    
\documentclass[11pt]{article}

    
    
    \usepackage[T1]{fontenc}
    % Nicer default font (+ math font) than Computer Modern for most use cases
    \usepackage{mathpazo}

    % Basic figure setup, for now with no caption control since it's done
    % automatically by Pandoc (which extracts ![](path) syntax from Markdown).
    \usepackage{graphicx}
    % We will generate all images so they have a width \maxwidth. This means
    % that they will get their normal width if they fit onto the page, but
    % are scaled down if they would overflow the margins.
    \makeatletter
    \def\maxwidth{\ifdim\Gin@nat@width>\linewidth\linewidth
    \else\Gin@nat@width\fi}
    \makeatother
    \let\Oldincludegraphics\includegraphics
    % Set max figure width to be 80% of text width, for now hardcoded.
    \renewcommand{\includegraphics}[1]{\Oldincludegraphics[width=.8\maxwidth]{#1}}
    % Ensure that by default, figures have no caption (until we provide a
    % proper Figure object with a Caption API and a way to capture that
    % in the conversion process - todo).
    \usepackage{caption}
    \DeclareCaptionLabelFormat{nolabel}{}
    \captionsetup{labelformat=nolabel}

    \usepackage{adjustbox} % Used to constrain images to a maximum size 
    \usepackage{xcolor} % Allow colors to be defined
    \usepackage{enumerate} % Needed for markdown enumerations to work
    \usepackage{geometry} % Used to adjust the document margins
    \usepackage{amsmath} % Equations
    \usepackage{amssymb} % Equations
    \usepackage{textcomp} % defines textquotesingle
    % Hack from http://tex.stackexchange.com/a/47451/13684:
    \AtBeginDocument{%
        \def\PYZsq{\textquotesingle}% Upright quotes in Pygmentized code
    }
    \usepackage{upquote} % Upright quotes for verbatim code
    \usepackage{eurosym} % defines \euro
    \usepackage[mathletters]{ucs} % Extended unicode (utf-8) support
    \usepackage[utf8x]{inputenc} % Allow utf-8 characters in the tex document
    \usepackage{fancyvrb} % verbatim replacement that allows latex
    \usepackage{grffile} % extends the file name processing of package graphics 
                         % to support a larger range 
    % The hyperref package gives us a pdf with properly built
    % internal navigation ('pdf bookmarks' for the table of contents,
    % internal cross-reference links, web links for URLs, etc.)
    \usepackage{hyperref}
    \usepackage{longtable} % longtable support required by pandoc >1.10
    \usepackage{booktabs}  % table support for pandoc > 1.12.2
    \usepackage[inline]{enumitem} % IRkernel/repr support (it uses the enumerate* environment)
    \usepackage[normalem]{ulem} % ulem is needed to support strikethroughs (\sout)
                                % normalem makes italics be italics, not underlines
    

    
    
    % Colors for the hyperref package
    \definecolor{urlcolor}{rgb}{0,.145,.698}
    \definecolor{linkcolor}{rgb}{.71,0.21,0.01}
    \definecolor{citecolor}{rgb}{.12,.54,.11}

    % ANSI colors
    \definecolor{ansi-black}{HTML}{3E424D}
    \definecolor{ansi-black-intense}{HTML}{282C36}
    \definecolor{ansi-red}{HTML}{E75C58}
    \definecolor{ansi-red-intense}{HTML}{B22B31}
    \definecolor{ansi-green}{HTML}{00A250}
    \definecolor{ansi-green-intense}{HTML}{007427}
    \definecolor{ansi-yellow}{HTML}{DDB62B}
    \definecolor{ansi-yellow-intense}{HTML}{B27D12}
    \definecolor{ansi-blue}{HTML}{208FFB}
    \definecolor{ansi-blue-intense}{HTML}{0065CA}
    \definecolor{ansi-magenta}{HTML}{D160C4}
    \definecolor{ansi-magenta-intense}{HTML}{A03196}
    \definecolor{ansi-cyan}{HTML}{60C6C8}
    \definecolor{ansi-cyan-intense}{HTML}{258F8F}
    \definecolor{ansi-white}{HTML}{C5C1B4}
    \definecolor{ansi-white-intense}{HTML}{A1A6B2}

    % commands and environments needed by pandoc snippets
    % extracted from the output of `pandoc -s`
    \providecommand{\tightlist}{%
      \setlength{\itemsep}{0pt}\setlength{\parskip}{0pt}}
    \DefineVerbatimEnvironment{Highlighting}{Verbatim}{commandchars=\\\{\}}
    % Add ',fontsize=\small' for more characters per line
    \newenvironment{Shaded}{}{}
    \newcommand{\KeywordTok}[1]{\textcolor[rgb]{0.00,0.44,0.13}{\textbf{{#1}}}}
    \newcommand{\DataTypeTok}[1]{\textcolor[rgb]{0.56,0.13,0.00}{{#1}}}
    \newcommand{\DecValTok}[1]{\textcolor[rgb]{0.25,0.63,0.44}{{#1}}}
    \newcommand{\BaseNTok}[1]{\textcolor[rgb]{0.25,0.63,0.44}{{#1}}}
    \newcommand{\FloatTok}[1]{\textcolor[rgb]{0.25,0.63,0.44}{{#1}}}
    \newcommand{\CharTok}[1]{\textcolor[rgb]{0.25,0.44,0.63}{{#1}}}
    \newcommand{\StringTok}[1]{\textcolor[rgb]{0.25,0.44,0.63}{{#1}}}
    \newcommand{\CommentTok}[1]{\textcolor[rgb]{0.38,0.63,0.69}{\textit{{#1}}}}
    \newcommand{\OtherTok}[1]{\textcolor[rgb]{0.00,0.44,0.13}{{#1}}}
    \newcommand{\AlertTok}[1]{\textcolor[rgb]{1.00,0.00,0.00}{\textbf{{#1}}}}
    \newcommand{\FunctionTok}[1]{\textcolor[rgb]{0.02,0.16,0.49}{{#1}}}
    \newcommand{\RegionMarkerTok}[1]{{#1}}
    \newcommand{\ErrorTok}[1]{\textcolor[rgb]{1.00,0.00,0.00}{\textbf{{#1}}}}
    \newcommand{\NormalTok}[1]{{#1}}
    
    % Additional commands for more recent versions of Pandoc
    \newcommand{\ConstantTok}[1]{\textcolor[rgb]{0.53,0.00,0.00}{{#1}}}
    \newcommand{\SpecialCharTok}[1]{\textcolor[rgb]{0.25,0.44,0.63}{{#1}}}
    \newcommand{\VerbatimStringTok}[1]{\textcolor[rgb]{0.25,0.44,0.63}{{#1}}}
    \newcommand{\SpecialStringTok}[1]{\textcolor[rgb]{0.73,0.40,0.53}{{#1}}}
    \newcommand{\ImportTok}[1]{{#1}}
    \newcommand{\DocumentationTok}[1]{\textcolor[rgb]{0.73,0.13,0.13}{\textit{{#1}}}}
    \newcommand{\AnnotationTok}[1]{\textcolor[rgb]{0.38,0.63,0.69}{\textbf{\textit{{#1}}}}}
    \newcommand{\CommentVarTok}[1]{\textcolor[rgb]{0.38,0.63,0.69}{\textbf{\textit{{#1}}}}}
    \newcommand{\VariableTok}[1]{\textcolor[rgb]{0.10,0.09,0.49}{{#1}}}
    \newcommand{\ControlFlowTok}[1]{\textcolor[rgb]{0.00,0.44,0.13}{\textbf{{#1}}}}
    \newcommand{\OperatorTok}[1]{\textcolor[rgb]{0.40,0.40,0.40}{{#1}}}
    \newcommand{\BuiltInTok}[1]{{#1}}
    \newcommand{\ExtensionTok}[1]{{#1}}
    \newcommand{\PreprocessorTok}[1]{\textcolor[rgb]{0.74,0.48,0.00}{{#1}}}
    \newcommand{\AttributeTok}[1]{\textcolor[rgb]{0.49,0.56,0.16}{{#1}}}
    \newcommand{\InformationTok}[1]{\textcolor[rgb]{0.38,0.63,0.69}{\textbf{\textit{{#1}}}}}
    \newcommand{\WarningTok}[1]{\textcolor[rgb]{0.38,0.63,0.69}{\textbf{\textit{{#1}}}}}
    
    
    % Define a nice break command that doesn't care if a line doesn't already
    % exist.
    \def\br{\hspace*{\fill} \\* }
    % Math Jax compatability definitions
    \def\gt{>}
    \def\lt{<}
    % Document parameters
    \title{Untitled}
    
    
    

    % Pygments definitions
    
\makeatletter
\def\PY@reset{\let\PY@it=\relax \let\PY@bf=\relax%
    \let\PY@ul=\relax \let\PY@tc=\relax%
    \let\PY@bc=\relax \let\PY@ff=\relax}
\def\PY@tok#1{\csname PY@tok@#1\endcsname}
\def\PY@toks#1+{\ifx\relax#1\empty\else%
    \PY@tok{#1}\expandafter\PY@toks\fi}
\def\PY@do#1{\PY@bc{\PY@tc{\PY@ul{%
    \PY@it{\PY@bf{\PY@ff{#1}}}}}}}
\def\PY#1#2{\PY@reset\PY@toks#1+\relax+\PY@do{#2}}

\expandafter\def\csname PY@tok@w\endcsname{\def\PY@tc##1{\textcolor[rgb]{0.73,0.73,0.73}{##1}}}
\expandafter\def\csname PY@tok@c\endcsname{\let\PY@it=\textit\def\PY@tc##1{\textcolor[rgb]{0.25,0.50,0.50}{##1}}}
\expandafter\def\csname PY@tok@cp\endcsname{\def\PY@tc##1{\textcolor[rgb]{0.74,0.48,0.00}{##1}}}
\expandafter\def\csname PY@tok@k\endcsname{\let\PY@bf=\textbf\def\PY@tc##1{\textcolor[rgb]{0.00,0.50,0.00}{##1}}}
\expandafter\def\csname PY@tok@kp\endcsname{\def\PY@tc##1{\textcolor[rgb]{0.00,0.50,0.00}{##1}}}
\expandafter\def\csname PY@tok@kt\endcsname{\def\PY@tc##1{\textcolor[rgb]{0.69,0.00,0.25}{##1}}}
\expandafter\def\csname PY@tok@o\endcsname{\def\PY@tc##1{\textcolor[rgb]{0.40,0.40,0.40}{##1}}}
\expandafter\def\csname PY@tok@ow\endcsname{\let\PY@bf=\textbf\def\PY@tc##1{\textcolor[rgb]{0.67,0.13,1.00}{##1}}}
\expandafter\def\csname PY@tok@nb\endcsname{\def\PY@tc##1{\textcolor[rgb]{0.00,0.50,0.00}{##1}}}
\expandafter\def\csname PY@tok@nf\endcsname{\def\PY@tc##1{\textcolor[rgb]{0.00,0.00,1.00}{##1}}}
\expandafter\def\csname PY@tok@nc\endcsname{\let\PY@bf=\textbf\def\PY@tc##1{\textcolor[rgb]{0.00,0.00,1.00}{##1}}}
\expandafter\def\csname PY@tok@nn\endcsname{\let\PY@bf=\textbf\def\PY@tc##1{\textcolor[rgb]{0.00,0.00,1.00}{##1}}}
\expandafter\def\csname PY@tok@ne\endcsname{\let\PY@bf=\textbf\def\PY@tc##1{\textcolor[rgb]{0.82,0.25,0.23}{##1}}}
\expandafter\def\csname PY@tok@nv\endcsname{\def\PY@tc##1{\textcolor[rgb]{0.10,0.09,0.49}{##1}}}
\expandafter\def\csname PY@tok@no\endcsname{\def\PY@tc##1{\textcolor[rgb]{0.53,0.00,0.00}{##1}}}
\expandafter\def\csname PY@tok@nl\endcsname{\def\PY@tc##1{\textcolor[rgb]{0.63,0.63,0.00}{##1}}}
\expandafter\def\csname PY@tok@ni\endcsname{\let\PY@bf=\textbf\def\PY@tc##1{\textcolor[rgb]{0.60,0.60,0.60}{##1}}}
\expandafter\def\csname PY@tok@na\endcsname{\def\PY@tc##1{\textcolor[rgb]{0.49,0.56,0.16}{##1}}}
\expandafter\def\csname PY@tok@nt\endcsname{\let\PY@bf=\textbf\def\PY@tc##1{\textcolor[rgb]{0.00,0.50,0.00}{##1}}}
\expandafter\def\csname PY@tok@nd\endcsname{\def\PY@tc##1{\textcolor[rgb]{0.67,0.13,1.00}{##1}}}
\expandafter\def\csname PY@tok@s\endcsname{\def\PY@tc##1{\textcolor[rgb]{0.73,0.13,0.13}{##1}}}
\expandafter\def\csname PY@tok@sd\endcsname{\let\PY@it=\textit\def\PY@tc##1{\textcolor[rgb]{0.73,0.13,0.13}{##1}}}
\expandafter\def\csname PY@tok@si\endcsname{\let\PY@bf=\textbf\def\PY@tc##1{\textcolor[rgb]{0.73,0.40,0.53}{##1}}}
\expandafter\def\csname PY@tok@se\endcsname{\let\PY@bf=\textbf\def\PY@tc##1{\textcolor[rgb]{0.73,0.40,0.13}{##1}}}
\expandafter\def\csname PY@tok@sr\endcsname{\def\PY@tc##1{\textcolor[rgb]{0.73,0.40,0.53}{##1}}}
\expandafter\def\csname PY@tok@ss\endcsname{\def\PY@tc##1{\textcolor[rgb]{0.10,0.09,0.49}{##1}}}
\expandafter\def\csname PY@tok@sx\endcsname{\def\PY@tc##1{\textcolor[rgb]{0.00,0.50,0.00}{##1}}}
\expandafter\def\csname PY@tok@m\endcsname{\def\PY@tc##1{\textcolor[rgb]{0.40,0.40,0.40}{##1}}}
\expandafter\def\csname PY@tok@gh\endcsname{\let\PY@bf=\textbf\def\PY@tc##1{\textcolor[rgb]{0.00,0.00,0.50}{##1}}}
\expandafter\def\csname PY@tok@gu\endcsname{\let\PY@bf=\textbf\def\PY@tc##1{\textcolor[rgb]{0.50,0.00,0.50}{##1}}}
\expandafter\def\csname PY@tok@gd\endcsname{\def\PY@tc##1{\textcolor[rgb]{0.63,0.00,0.00}{##1}}}
\expandafter\def\csname PY@tok@gi\endcsname{\def\PY@tc##1{\textcolor[rgb]{0.00,0.63,0.00}{##1}}}
\expandafter\def\csname PY@tok@gr\endcsname{\def\PY@tc##1{\textcolor[rgb]{1.00,0.00,0.00}{##1}}}
\expandafter\def\csname PY@tok@ge\endcsname{\let\PY@it=\textit}
\expandafter\def\csname PY@tok@gs\endcsname{\let\PY@bf=\textbf}
\expandafter\def\csname PY@tok@gp\endcsname{\let\PY@bf=\textbf\def\PY@tc##1{\textcolor[rgb]{0.00,0.00,0.50}{##1}}}
\expandafter\def\csname PY@tok@go\endcsname{\def\PY@tc##1{\textcolor[rgb]{0.53,0.53,0.53}{##1}}}
\expandafter\def\csname PY@tok@gt\endcsname{\def\PY@tc##1{\textcolor[rgb]{0.00,0.27,0.87}{##1}}}
\expandafter\def\csname PY@tok@err\endcsname{\def\PY@bc##1{\setlength{\fboxsep}{0pt}\fcolorbox[rgb]{1.00,0.00,0.00}{1,1,1}{\strut ##1}}}
\expandafter\def\csname PY@tok@kc\endcsname{\let\PY@bf=\textbf\def\PY@tc##1{\textcolor[rgb]{0.00,0.50,0.00}{##1}}}
\expandafter\def\csname PY@tok@kd\endcsname{\let\PY@bf=\textbf\def\PY@tc##1{\textcolor[rgb]{0.00,0.50,0.00}{##1}}}
\expandafter\def\csname PY@tok@kn\endcsname{\let\PY@bf=\textbf\def\PY@tc##1{\textcolor[rgb]{0.00,0.50,0.00}{##1}}}
\expandafter\def\csname PY@tok@kr\endcsname{\let\PY@bf=\textbf\def\PY@tc##1{\textcolor[rgb]{0.00,0.50,0.00}{##1}}}
\expandafter\def\csname PY@tok@bp\endcsname{\def\PY@tc##1{\textcolor[rgb]{0.00,0.50,0.00}{##1}}}
\expandafter\def\csname PY@tok@fm\endcsname{\def\PY@tc##1{\textcolor[rgb]{0.00,0.00,1.00}{##1}}}
\expandafter\def\csname PY@tok@vc\endcsname{\def\PY@tc##1{\textcolor[rgb]{0.10,0.09,0.49}{##1}}}
\expandafter\def\csname PY@tok@vg\endcsname{\def\PY@tc##1{\textcolor[rgb]{0.10,0.09,0.49}{##1}}}
\expandafter\def\csname PY@tok@vi\endcsname{\def\PY@tc##1{\textcolor[rgb]{0.10,0.09,0.49}{##1}}}
\expandafter\def\csname PY@tok@vm\endcsname{\def\PY@tc##1{\textcolor[rgb]{0.10,0.09,0.49}{##1}}}
\expandafter\def\csname PY@tok@sa\endcsname{\def\PY@tc##1{\textcolor[rgb]{0.73,0.13,0.13}{##1}}}
\expandafter\def\csname PY@tok@sb\endcsname{\def\PY@tc##1{\textcolor[rgb]{0.73,0.13,0.13}{##1}}}
\expandafter\def\csname PY@tok@sc\endcsname{\def\PY@tc##1{\textcolor[rgb]{0.73,0.13,0.13}{##1}}}
\expandafter\def\csname PY@tok@dl\endcsname{\def\PY@tc##1{\textcolor[rgb]{0.73,0.13,0.13}{##1}}}
\expandafter\def\csname PY@tok@s2\endcsname{\def\PY@tc##1{\textcolor[rgb]{0.73,0.13,0.13}{##1}}}
\expandafter\def\csname PY@tok@sh\endcsname{\def\PY@tc##1{\textcolor[rgb]{0.73,0.13,0.13}{##1}}}
\expandafter\def\csname PY@tok@s1\endcsname{\def\PY@tc##1{\textcolor[rgb]{0.73,0.13,0.13}{##1}}}
\expandafter\def\csname PY@tok@mb\endcsname{\def\PY@tc##1{\textcolor[rgb]{0.40,0.40,0.40}{##1}}}
\expandafter\def\csname PY@tok@mf\endcsname{\def\PY@tc##1{\textcolor[rgb]{0.40,0.40,0.40}{##1}}}
\expandafter\def\csname PY@tok@mh\endcsname{\def\PY@tc##1{\textcolor[rgb]{0.40,0.40,0.40}{##1}}}
\expandafter\def\csname PY@tok@mi\endcsname{\def\PY@tc##1{\textcolor[rgb]{0.40,0.40,0.40}{##1}}}
\expandafter\def\csname PY@tok@il\endcsname{\def\PY@tc##1{\textcolor[rgb]{0.40,0.40,0.40}{##1}}}
\expandafter\def\csname PY@tok@mo\endcsname{\def\PY@tc##1{\textcolor[rgb]{0.40,0.40,0.40}{##1}}}
\expandafter\def\csname PY@tok@ch\endcsname{\let\PY@it=\textit\def\PY@tc##1{\textcolor[rgb]{0.25,0.50,0.50}{##1}}}
\expandafter\def\csname PY@tok@cm\endcsname{\let\PY@it=\textit\def\PY@tc##1{\textcolor[rgb]{0.25,0.50,0.50}{##1}}}
\expandafter\def\csname PY@tok@cpf\endcsname{\let\PY@it=\textit\def\PY@tc##1{\textcolor[rgb]{0.25,0.50,0.50}{##1}}}
\expandafter\def\csname PY@tok@c1\endcsname{\let\PY@it=\textit\def\PY@tc##1{\textcolor[rgb]{0.25,0.50,0.50}{##1}}}
\expandafter\def\csname PY@tok@cs\endcsname{\let\PY@it=\textit\def\PY@tc##1{\textcolor[rgb]{0.25,0.50,0.50}{##1}}}

\def\PYZbs{\char`\\}
\def\PYZus{\char`\_}
\def\PYZob{\char`\{}
\def\PYZcb{\char`\}}
\def\PYZca{\char`\^}
\def\PYZam{\char`\&}
\def\PYZlt{\char`\<}
\def\PYZgt{\char`\>}
\def\PYZsh{\char`\#}
\def\PYZpc{\char`\%}
\def\PYZdl{\char`\$}
\def\PYZhy{\char`\-}
\def\PYZsq{\char`\'}
\def\PYZdq{\char`\"}
\def\PYZti{\char`\~}
% for compatibility with earlier versions
\def\PYZat{@}
\def\PYZlb{[}
\def\PYZrb{]}
\makeatother


    % Exact colors from NB
    \definecolor{incolor}{rgb}{0.0, 0.0, 0.5}
    \definecolor{outcolor}{rgb}{0.545, 0.0, 0.0}



    
    % Prevent overflowing lines due to hard-to-break entities
    \sloppy 
    % Setup hyperref package
    \hypersetup{
      breaklinks=true,  % so long urls are correctly broken across lines
      colorlinks=true,
      urlcolor=urlcolor,
      linkcolor=linkcolor,
      citecolor=citecolor,
      }
    % Slightly bigger margins than the latex defaults
    
    \geometry{verbose,tmargin=1in,bmargin=1in,lmargin=1in,rmargin=1in}
    
    

    \begin{document}
    
    
    \maketitle
    
    

    
    \#

{Battle of the contracts: Employment in the modern economy}

    \begin{longtable}[]{@{}ll@{}}
\toprule
Name & ANR\tabularnewline
\midrule
\endhead
Angel Rousev & 582598\tabularnewline
\bottomrule
\end{longtable}

    \#\#

{Question}

\textbf{\emph{Do temporary contracts harm employment?}}

    \#\#

{Motivation}

\begin{verbatim}
EAnyone between the ages of 15 and 65, is considered to be part of the national labor force. On employment, they face economic cycles and different legal amendments, which contribute to the employment figures and the state of employment contracts.  [The latest trend](https://observatoriosociallacaixa.org/documents/22890/80956/ART6_ENG_Graph1.jpg/3605c46d-3bff-4765-bc26-9d8a438fa156?t=1473783305328 "Wordwide Laborcontract Trend") shows a rise in flexible/temporary contracts. This in term causes the number of permanent contracts to go down, which deceases stability in the job market.  Does this help society to will this prevent more and more employees to garner any form of permanent contract, thus hurting employment?  </div>
\end{verbatim}

    \#\#

{Brief introduction}

\begin{verbatim}
In the first part, the impact on the labor market due to the change in contracts will be examined. This will cover the institution and the effects on policy. Second will be the latest situation in The Netherlands with regard to the change in contracts. In the third part, the motion of a new policy, will be discussed. This will include the state of the motion and the residuum of that motion.          </div>
\end{verbatim}

    \#\#

{Data}

The transformed national data will be recovered from
\href{http://statline.cbs.nl/StatWeb/publication/?VW=T\&DM=SLNL\&PA=70072ned\&D1=0-118\&D2=0,12\&D3=14-15\&HD=100914-1525\&HDR=T\&STB=G1,G2}{CBS's
Statline} The notebook will further cover theory and policy implications
derived from the data and trends within the labor market

    \#\#

{Theory of institution}

This part will cover the theory about the EPL, the Employment Protection
Legislation 1*

The EPL is a system of procedure which protects employees from premature
job-loss, i.e. getting fired. It covers temporary/flexible contracts,
permanent contracts and collective dismissal in a macro causal domain.
The EPL is measured as an indication which is a sum of the weighted
averages of said indicators. For example, in the case of severance pay
as shown below.

    \[Severance Pay Indicator = {(9Months Tenure + 4Years Tenure + 20Years Tenure)/3)}\]
\[Severance Pay Indicator = {(a + b + c)/3)}\]
\[{With ((a,b,c)| ϵ  0,1,2,3,4,5,6))}\] Click
\href{https://www.oecd.org/els/emp/EPL-Document-LAC-Methodology-ENG.pdf}{here}
to see the full list of the official EPL indicators.

    In figure 1, you can see the Netherlands having a higher than average
permanent contract rate. This provides employees with job security. In
the next part, the decline in permanent contracts will be shown. This
trend has financial benefits for the employers and takes bargaining
power away from employees. The shortcoming of the EPL, besides the fact
that it is based on a survey, states nothing about actual enforcement of
a dismissal.

    {Figure 1}

\includegraphics{https://image.ibb.co/mCPxMk/11.png}

    Figure 2 shows a negative correlation between the transition or
temporary/flexible workers to a permanent contract. Figure 3 further
illustrates the GDP-relationship as it pertains to the labor contracts,
which does not match previous EPL-statistics, meaning that the economy
is forcing the rise of temporary/flexible contracts.

    {Figure 2}

\includegraphics{https://image.ibb.co/ip2C45/12.png}

    {Figure 3}

\includegraphics{https://image.ibb.co/jbVtBk/13.png}

    There is also a phenomenon called the Honeymoon effect, which states
that some flexibility serves as a buffer of the temporary/flexible
contracts to replace permanent contracts. The next part of the notebook
will focus on the employment on a national level and will display the
evolution of the labor contracts within The Netherlands

    \#\#

{Empirical Evidence In The Netherlands}

This part will illustrate the labor contracts and the labor market in
The Netherlands. Figure 3 shows the distribution of labor contracts in
The Netherlands. Figure 4 represents the dataset as presented from the
CBS.

\begin{verbatim}
</div>
\end{verbatim}

    {Figure 3}

\includegraphics{http://recruitmentmatters.nl/wp-content/uploads/2013/10/image93.png}

    {Figure 4}

    \begin{Verbatim}[commandchars=\\\{\}]
{\color{incolor}In [{\color{incolor}98}]:} \PY{k+kn}{import} \PY{n+nn}{pandas} \PY{k}{as} \PY{n+nn}{pd}
         \PY{k+kn}{import} \PY{n+nn}{matplotlib}\PY{n+nn}{.}\PY{n+nn}{pyplot} \PY{k}{as} \PY{n+nn}{plt}
         \PY{k+kn}{import} \PY{n+nn}{matplotlib}\PY{n+nn}{.}\PY{n+nn}{dates} \PY{k}{as} \PY{n+nn}{mdates}
         \PY{o}{\PYZpc{}}\PY{k}{matplotlib} inline
         
         \PY{n}{data} \PY{o}{=} \PY{n}{pd}\PY{o}{.}\PY{n}{read\PYZus{}csv}\PY{p}{(}\PY{l+s+s1}{\PYZsq{}}\PY{l+s+s1}{datasets/datasets/12.csv}\PY{l+s+s1}{\PYZsq{}}\PY{p}{,} \PY{n}{usecols}\PY{o}{=}\PY{p}{[}\PY{l+s+s1}{\PYZsq{}}\PY{l+s+s1}{Periode}\PY{l+s+s1}{\PYZsq{}}\PY{p}{,}\PY{l+s+s1}{\PYZsq{}}\PY{l+s+s1}{Toaal Werzame Personen}\PY{l+s+s1}{\PYZsq{}}\PY{p}{,}\PY{l+s+s1}{\PYZsq{}}\PY{l+s+s1}{Werknemer}\PY{l+s+s1}{\PYZsq{}}\PY{p}{,}\PY{l+s+s1}{\PYZsq{}}\PY{l+s+s1}{Werknemer met vaste arbeidsrelatie}\PY{l+s+s1}{\PYZsq{}}\PY{p}{,}\PY{l+s+s1}{\PYZsq{}}\PY{l+s+s1}{Werknemer met flexibele arbeidsrelatie}\PY{l+s+s1}{\PYZsq{}}\PY{p}{]}\PY{p}{,} \PY{n}{parse\PYZus{}dates}\PY{o}{=}\PY{p}{[}\PY{l+s+s1}{\PYZsq{}}\PY{l+s+s1}{Periode}\PY{l+s+s1}{\PYZsq{}}\PY{p}{]}\PY{p}{)}
         
         \PY{n}{data}\PY{o}{.}\PY{n}{set\PYZus{}index}\PY{p}{(}\PY{l+s+s1}{\PYZsq{}}\PY{l+s+s1}{Periode}\PY{l+s+s1}{\PYZsq{}}\PY{p}{,}\PY{n}{inplace}\PY{o}{=}\PY{k+kc}{True}\PY{p}{)}
         
         \PY{n}{fig}\PY{p}{,} \PY{n}{ax} \PY{o}{=} \PY{n}{plt}\PY{o}{.}\PY{n}{subplots}\PY{p}{(}\PY{n}{figsize}\PY{o}{=}\PY{p}{(}\PY{l+m+mi}{12}\PY{p}{,}\PY{l+m+mi}{18}\PY{p}{)}\PY{p}{)}
         \PY{n}{data}\PY{o}{.}\PY{n}{plot}\PY{p}{(}\PY{n}{kind}\PY{o}{=}\PY{l+s+s1}{\PYZsq{}}\PY{l+s+s1}{bar}\PY{l+s+s1}{\PYZsq{}}\PY{p}{,} \PY{n}{ax}\PY{o}{=}\PY{n}{ax}\PY{p}{)}
\end{Verbatim}

            \begin{Verbatim}[commandchars=\\\{\}]
{\color{outcolor}Out[{\color{outcolor}98}]:} <matplotlib.axes.\_subplots.AxesSubplot at 0x23d16e3de48>
\end{Verbatim}
        
    \begin{center}
    \adjustimage{max size={0.9\linewidth}{0.9\paperheight}}{output_17_1.png}
    \end{center}
    { \hspace*{\fill} \\}
    
    \begin{verbatim}
           The main law in The Netherlands that covers the labor contracts is ‘’De Wet Flexibiliteit en Zekerheid’’, also known as De Flexwet. Introduced in 1999, this law was meant to balance flexibility and job-security, thus creating stability within the power relationship of the employer and the employee. The law states that there can only be three temporary contracts offered within a span of three years. After the three-year period, the temporary/flexible contract automatically becomes a permanent contract.
\end{verbatim}

2015 was the year where new additions to De Flexwet were added. These
were created to fight the increasing gap between both labor contracts.
The law now stated that the change from a temporary/flexible contract to
a permanent contract could be prevented if there was a period of six
month in-between both contracts. Another key addition was the
transformation from a temporary/flexible contract to a permanent
contract, if an individual worked over 2 years. This would become a
major issue, due to the countermeasure that was, firing the employee
before the 2-year threshold was reached. The CNV counted 200 cases in
2015 of these scenarios in The Netherlands. The same held true for the
six-month threshold. Minister Asscher proposed a second change in 2015
where a `'transitievergoeding'' was granted to employees, with contract
terminations after the 2-year threshold. This would entail a one-third
payment of their ex-wage and one-half if total tenure was under ten
years, with a max of 76000 euros. A full wage would be paid for
employees over 50, no matter the employment duration. Thus, De Flexwet
is having the opposite effect and is damaging employees'' chances. This
rise can be seen in figure 3 and in the following CBS dataset.

    \#\#

{Policy Analysis}

The EPL seems to cause work-arounds to occur more often and make
employers focus on a new time stamp for their employees' termination.
Thus, it becomes clear that further and better protection is needed to
cover the employees against these measures. In The Netherlands, income
is guaranteed in some form when the worst scenario tends to occur, but
job security is not tackled in the slightest. This would probably cause
a rift between the principal and agent in this scenario, leading to less
effort and investment from the now, anticipation employee. A scenario
could also form where the buffer and the seasonal and cyclical changes,
return to the original state. This would make it optimal for employers
not to fire their employees as it will save several costs, including
searching costs and loss on investment.

An increase in the EPL causes fewer hiring's. One of the countries that
has countered this problem in a way was Austria, which in 2003 replaced
their dismissal payment system, which was based on tenure. Now employees
have to pay a percentage of the payroll, which becomes a savings-account
for the employee. This procedure ends when the contract comes to an end.
This will serve as an income transfer measure when the contract would be
terminated. If no such scenario would occur, the funds would transfer to
the retirement fund of said individual.

A great policy would be a form of tenure, where employees experience
trust and safety in combination with the Austrian legislature amendment
of 2003. This form of EPL has less negative effects to ones with a
permanent contract and will clear up the 3-year threshold as seen in The
Netherlands. Partial reforms could provoke create labor market
dualities, thus a full reform would be beneficial.

    \#\#

{Conclusion}

A change in policy is needed. Europe and The Netherlands in particular
is facing job security threats from institutional loopholes. Thresholds
are being broken and even retirement policies have been subjugated to
these offenses (A diff-in-diff analysis of the effects on pension when a
certain birthrate was applied as a treatment threshold). In order to
make these preventive events happen a similar policy like in Austria has
to be implemented, which creates a form of insurance for both parties.
Governments fail to anticipate these events and employees should be
trapped in a vortex of temporary/flexible contracts. One has the right
to build up their career and be invested in the company in which they
function. I'm afraid that training and investment will become
meaningless and a job will become just a job for most.

\begin{verbatim}
</div>
\end{verbatim}

     1*

{ Employment Practices Liability is an area of United States law that
deals with wrongful termination, sexual harassment, discrimination,
invasion of privacy, false imprisonment, breach of contract, emotional
distress, and wage and hour law violations. Employment Practices
Liability is part of professional liability. }

    \#\#

{References}

Jenifer Ruiz-Valenzuela, (2016, September) Temporality, loss of work and
educational performance. Centre for Economic Performance, London School
of Economics. Retrieved from
https://observatoriosociallacaixa.org/en/article/-/asset\_publisher/ATai9MyKZiYq/content/el-impacto-de-la-temporalidad-y-la-perdida-de-trabajo-parental-en-el-rendimiento-educativo-de-los-hijos/pop\_up
Jabob, A. (2013). `'The Effect of Employment Protection on Teacher
Effort''. Journal of Labor Ecocomics, 31(4): 727-761

Schnalzenberger, M. and Winter-Ebme, R. `'Layoff tax and employment of
the elderly''. Labor Economics, 16(6): 618-624

Olsson, M. (2009). `'Employment protection and sickness absence''. Labor
Economics, 16(2): 208-214 Employment protection legislation. (2015)
Retrieved from
http://ec.europa.eu/europe2020/pdf/themes/25\_employment\_protection\_legislation\_02.pd

Wijzigingen Flexwet 2015. (2015) Retrieved from
http://www.payrolltoday.nl/uploads/pdf/WIJZIGINGEN\_FLEXWET\_2015.pdf

Drees, M. (2013) De flexibele schil in beeld. Retireved from
http://recruitmentmatters.nl/2013/10/24/de-flexibele-schil-in-beeld-2/

Driessen, M. and Lautenbach, H. (2012, December 12). Minder Werknemers
Met Een Vast Conctract. CBS. Retrieved from
https://www.cbs.nl/nl-nl/nieuws/2012/50/minder-werknemers-met-een-vast-contract

Limmen, M. (2015, May 6) Werkgevers omzeilen www door flexkrachten te
ontslaan. CNV. Retrieved from
https://www.cnv.nl/actueel/nieuws/nieuwsdetail/werkgevers-omzeilen-wwz-door-flexkrachten-te-ontslaan/?L=0\%2F\&cHash=536a52bf52d90cf189cce000cee110ae

Driessen, M. (2015, November 13) Positie werkkring en arbeidsduur:
aantal vaste werknemers niet verder gedaald. CBS. Retrieved from
https://www.cbs.nl/nl-nl/achtergrond/2015/46/positie-werkkring-en-arbeidsduur-aantal-vaste-werknemers-niet-verder-gedaald

OECD (2004) OECD Employment Outlook 2004. ISBN: 9789264108134 (PDF)
;9789264108127(print). DOI: 10.1787/empl\_outlook-2004-en Retrieved from
http://www.oecd-ilibrary.org/employment/oecd-employment-outlook-
2004\_empl\_outlook-2004-en Regionale kerncijfers Nederland. (2017)
Retrieved from
http://statline.cbs.nl/Statweb/publication/?DM=SLNL\&PA=70072ned\&D1=0-118\&D2=0,12\&D3=14-15\&HDR=T\&STB=G1,G2\&VW=C

    \#\#

{Code}

    \begin{Verbatim}[commandchars=\\\{\}]
{\color{incolor}In [{\color{incolor}69}]:} \PY{c+c1}{\PYZsh{} import modules}
         \PY{k+kn}{import} \PY{n+nn}{pandas} \PY{k}{as} \PY{n+nn}{pd}
\end{Verbatim}

    \begin{Verbatim}[commandchars=\\\{\}]
{\color{incolor}In [{\color{incolor}70}]:} \PY{c+c1}{\PYZsh{} Import the excel file and call it xls\PYZus{}file}
         \PY{n}{xls\PYZus{}file} \PY{o}{=} \PY{n}{pd}\PY{o}{.}\PY{n}{ExcelFile}\PY{p}{(}\PY{l+s+s1}{\PYZsq{}}\PY{l+s+s1}{datasets/datasets/examples.xls}\PY{l+s+s1}{\PYZsq{}}\PY{p}{)}
         \PY{n}{xls\PYZus{}file}
\end{Verbatim}

            \begin{Verbatim}[commandchars=\\\{\}]
{\color{outcolor}Out[{\color{outcolor}70}]:} <pandas.io.excel.ExcelFile at 0x23d16843cf8>
\end{Verbatim}
        
    \begin{Verbatim}[commandchars=\\\{\}]
{\color{incolor}In [{\color{incolor}71}]:} \PY{c+c1}{\PYZsh{} View the excel file\PYZsq{}s sheet names}
         \PY{n}{xls\PYZus{}file}\PY{o}{.}\PY{n}{sheet\PYZus{}names}
\end{Verbatim}

            \begin{Verbatim}[commandchars=\\\{\}]
{\color{outcolor}Out[{\color{outcolor}71}]:} ['Werkzame\_beroepsbev.\_3008172034', 'Omschrijving']
\end{Verbatim}
        
    \begin{Verbatim}[commandchars=\\\{\}]
{\color{incolor}In [{\color{incolor}72}]:} \PY{c+c1}{\PYZsh{} Load the xls file\PYZsq{}s Sheet1 as a dataframe}
         \PY{n}{df} \PY{o}{=} \PY{n}{xls\PYZus{}file}\PY{o}{.}\PY{n}{parse}\PY{p}{(}\PY{l+s+s1}{\PYZsq{}}\PY{l+s+s1}{Werkzame\PYZus{}beroepsbev.\PYZus{}3008172034}\PY{l+s+s1}{\PYZsq{}}\PY{p}{)}
         \PY{n}{df}
\end{Verbatim}

            \begin{Verbatim}[commandchars=\\\{\}]
{\color{outcolor}Out[{\color{outcolor}72}]:}    Werkzame beroepsbevolking; positie in de werkkring  \textbackslash{}
         0                                                 NaN   
         1                                                 NaN   
         2                                                 NaN   
         3                                            Geslacht   
         4                            Totaal mannen en vrouwen   
         5                            Totaal mannen en vrouwen   
         6                            Totaal mannen en vrouwen   
         7                            Totaal mannen en vrouwen   
         8                                              Mannen   
         9                                              Mannen   
         10                                             Mannen   
         11                                             Mannen   
         12                                            Vrouwen   
         13                                            Vrouwen   
         14                                            Vrouwen   
         15                                            Vrouwen   
         
                          Unnamed: 1                 Unnamed: 2  \textbackslash{}
         0               Onderwerpen  Werkzame beroepsbevolking   
         1         Persoonskenmerken            Totaal personen   
         2   Positie in de werkkring                     Totaal   
         3                  Perioden                    x 1 000   
         4          2016 1e kwartaal                       8287   
         5          2016 2e kwartaal                       8386   
         6          2016 3e kwartaal                       8461   
         7          2016 4e kwartaal                       8478   
         8          2016 1e kwartaal                       4464   
         9          2016 2e kwartaal                       4502   
         10         2016 3e kwartaal                       4541   
         11         2016 4e kwartaal                       4553   
         12         2016 1e kwartaal                       3823   
         13         2016 2e kwartaal                       3884   
         14         2016 3e kwartaal                       3920   
         15         2016 4e kwartaal                       3925   
         
                            Unnamed: 3                          Unnamed: 4  \textbackslash{}
         0   Werkzame beroepsbevolking           Werkzame beroepsbevolking   
         1             Totaal personen                     Totaal personen   
         2                   Werknemer  Werknemer met vaste arbeidsrelatie   
         3                     x 1 000                             x 1 000   
         4                        6894                                5146   
         5                        7000                                5163   
         6                        7058                                5169   
         7                        7047                                5155   
         8                        3584                                2733   
         9                        3615                                2729   
         10                       3646                                2713   
         11                       3655                                2731   
         12                       3310                                2413   
         13                       3385                                2435   
         14                       3412                                2456   
         15                       3392                                2424   
         
                                         Unnamed: 5  \textbackslash{}
         0                Werkzame beroepsbevolking   
         1                          Totaal personen   
         2   Werknemer met flexibele arbeidsrelatie   
         3                                  x 1 000   
         4                                     1748   
         5                                     1836   
         6                                     1889   
         7                                     1892   
         8                                      851   
         9                                      886   
         10                                     933   
         11                                     924   
         12                                     897   
         13                                     950   
         14                                     956   
         15                                     968   
         
                                     Unnamed: 6                              Unnamed: 7  
         0            Werkzame beroepsbevolking               Werkzame beroepsbevolking  
         1                      Totaal personen                         Totaal personen  
         2   Werknemer met vaste arbeidsrelatie  Werknemer met flexibele arbeidsrelatie  
         3                                    \%                                       \%  
         4                             0.746446                                0.339681  
         5                             0.737571                                0.355607  
         6                              0.73236                                0.365448  
         7                             0.731517                                0.367022  
         8                             0.762556                                0.311379  
         9                              0.75491                                0.324661  
         10                            0.744103                                  0.3439  
         11                            0.747196                                0.338338  
         12                            0.729003                                0.371736  
         13                             0.71935                                0.390144  
         14                            0.719812                                0.389251  
         15                            0.714623                                 0.39934  
\end{Verbatim}
        
    \begin{Verbatim}[commandchars=\\\{\}]
{\color{incolor}In [{\color{incolor}73}]:} \PY{c+c1}{\PYZsh{} Figured out the Excel lay\PYZhy{}out would be problamatic. I then corrected for the upper cells and ran the CSV\PYZhy{}file.}
\end{Verbatim}

    \begin{Verbatim}[commandchars=\\\{\}]
{\color{incolor}In [{\color{incolor}74}]:} \PY{k+kn}{import} \PY{n+nn}{pandas} \PY{k}{as} \PY{n+nn}{pd}
         \PY{n}{df1}\PY{o}{=}\PY{n}{pd}\PY{o}{.}\PY{n}{read\PYZus{}csv}\PY{p}{(}\PY{l+s+s1}{\PYZsq{}}\PY{l+s+s1}{datasets/datasets/11.csv}\PY{l+s+s1}{\PYZsq{}}\PY{p}{)}
         \PY{n+nb}{print}\PY{p}{(}\PY{n}{df1}\PY{p}{)}
\end{Verbatim}

    \begin{Verbatim}[commandchars=\\\{\}]
                    Geslacht           Periode  Toaal Werzame Personen  \textbackslash{}
0   Totaal mannen en vrouwen  2016 1e kwartaal                    8287   
1   Totaal mannen en vrouwen  2016 2e kwartaal                    8386   
2   Totaal mannen en vrouwen  2016 3e kwartaal                    8461   
3   Totaal mannen en vrouwen  2016 4e kwartaal                    8478   
4                     Mannen  2016 1e kwartaal                    4464   
5                     Mannen  2016 2e kwartaal                    4502   
6                     Mannen  2016 3e kwartaal                    4541   
7                     Mannen  2016 4e kwartaal                    4553   
8                    Vrouwen  2016 1e kwartaal                    3823   
9                    Vrouwen  2016 2e kwartaal                    3884   
10                   Vrouwen  2016 3e kwartaal                    3920   
11                   Vrouwen  2016 4e kwartaal                    3925   

    Werknemer  Werknemer met vaste arbeidsrelatie  \textbackslash{}
0        6894                                5146   
1        7000                                5163   
2        7058                                5169   
3        7047                                5155   
4        3584                                2733   
5        3615                                2729   
6        3646                                2713   
7        3655                                2731   
8        3310                                2413   
9        3385                                2435   
10       3412                                2456   
11       3392                                2424   

    Werknemer met flexibele arbeidsrelatie  \textbackslash{}
0                                     1748   
1                                     1836   
2                                     1889   
3                                     1892   
4                                      851   
5                                      886   
6                                      933   
7                                      924   
8                                      897   
9                                      950   
10                                     956   
11                                     968   

   Percentage werknemer met  vaste arbeidsrelatie  \textbackslash{}
0                                          74.64\%   
1                                          73.76\%   
2                                          73.24\%   
3                                          73.15\%   
4                                          76.26\%   
5                                          75.49\%   
6                                          74.41\%   
7                                          74.72\%   
8                                          72.90\%   
9                                          71.94\%   
10                                         71.98\%   
11                                         71.46\%   

   Percentage werknemer met flexibele arbeidsrelatie  
0                                             33.97\%  
1                                             35.56\%  
2                                             36.54\%  
3                                             36.70\%  
4                                             31.14\%  
5                                             32.47\%  
6                                             34.39\%  
7                                             33.83\%  
8                                             37.17\%  
9                                             39.01\%  
10                                            38.93\%  
11                                            39.93\%  

    \end{Verbatim}

    \begin{Verbatim}[commandchars=\\\{\}]
{\color{incolor}In [{\color{incolor}75}]:} \PY{c+c1}{\PYZsh{} Including A label }
\end{Verbatim}

    \[x1000     =  {Totaal werkzame personen}\]

\[x1000.1   =  {Werknemer}\]
\[x1000.2   =  {Werknemer met vaste arbeidsrelatie}\]

\[x1000.3   =  {Werknemer met flexibele arbeidsrelatie}\]
\[Percentage   =  {Percentage werknemer met vaste arbeidsrelatie}\]
\[Percentage1   =  {Percentage werknemer met flexibele arbeidsrelatie}\]

    \begin{Verbatim}[commandchars=\\\{\}]
{\color{incolor}In [{\color{incolor}76}]:} \PY{n}{df1}
\end{Verbatim}

            \begin{Verbatim}[commandchars=\\\{\}]
{\color{outcolor}Out[{\color{outcolor}76}]:}                     Geslacht           Periode  Toaal Werzame Personen  \textbackslash{}
         0   Totaal mannen en vrouwen  2016 1e kwartaal                    8287   
         1   Totaal mannen en vrouwen  2016 2e kwartaal                    8386   
         2   Totaal mannen en vrouwen  2016 3e kwartaal                    8461   
         3   Totaal mannen en vrouwen  2016 4e kwartaal                    8478   
         4                     Mannen  2016 1e kwartaal                    4464   
         5                     Mannen  2016 2e kwartaal                    4502   
         6                     Mannen  2016 3e kwartaal                    4541   
         7                     Mannen  2016 4e kwartaal                    4553   
         8                    Vrouwen  2016 1e kwartaal                    3823   
         9                    Vrouwen  2016 2e kwartaal                    3884   
         10                   Vrouwen  2016 3e kwartaal                    3920   
         11                   Vrouwen  2016 4e kwartaal                    3925   
         
             Werknemer  Werknemer met vaste arbeidsrelatie  \textbackslash{}
         0        6894                                5146   
         1        7000                                5163   
         2        7058                                5169   
         3        7047                                5155   
         4        3584                                2733   
         5        3615                                2729   
         6        3646                                2713   
         7        3655                                2731   
         8        3310                                2413   
         9        3385                                2435   
         10       3412                                2456   
         11       3392                                2424   
         
             Werknemer met flexibele arbeidsrelatie  \textbackslash{}
         0                                     1748   
         1                                     1836   
         2                                     1889   
         3                                     1892   
         4                                      851   
         5                                      886   
         6                                      933   
         7                                      924   
         8                                      897   
         9                                      950   
         10                                     956   
         11                                     968   
         
            Percentage werknemer met  vaste arbeidsrelatie  \textbackslash{}
         0                                          74.64\%   
         1                                          73.76\%   
         2                                          73.24\%   
         3                                          73.15\%   
         4                                          76.26\%   
         5                                          75.49\%   
         6                                          74.41\%   
         7                                          74.72\%   
         8                                          72.90\%   
         9                                          71.94\%   
         10                                         71.98\%   
         11                                         71.46\%   
         
            Percentage werknemer met flexibele arbeidsrelatie  
         0                                             33.97\%  
         1                                             35.56\%  
         2                                             36.54\%  
         3                                             36.70\%  
         4                                             31.14\%  
         5                                             32.47\%  
         6                                             34.39\%  
         7                                             33.83\%  
         8                                             37.17\%  
         9                                             39.01\%  
         10                                            38.93\%  
         11                                            39.93\%  
\end{Verbatim}
        
    \begin{Verbatim}[commandchars=\\\{\}]
{\color{incolor}In [{\color{incolor}77}]:} \PY{c+c1}{\PYZsh{} Including labels in a text file of the csv as an alteration}
\end{Verbatim}

    \begin{Verbatim}[commandchars=\\\{\}]
{\color{incolor}In [{\color{incolor}78}]:} \PY{k+kn}{import} \PY{n+nn}{pandas} \PY{k}{as} \PY{n+nn}{pd}
         \PY{n}{df1}\PY{o}{=}\PY{n}{pd}\PY{o}{.}\PY{n}{read\PYZus{}csv}\PY{p}{(}\PY{l+s+s1}{\PYZsq{}}\PY{l+s+s1}{datasets/datasets/11.csv}\PY{l+s+s1}{\PYZsq{}}\PY{p}{)}
         \PY{n+nb}{print}\PY{p}{(}\PY{n}{df1}\PY{p}{)}
\end{Verbatim}

    \begin{Verbatim}[commandchars=\\\{\}]
                    Geslacht           Periode  Toaal Werzame Personen  \textbackslash{}
0   Totaal mannen en vrouwen  2016 1e kwartaal                    8287   
1   Totaal mannen en vrouwen  2016 2e kwartaal                    8386   
2   Totaal mannen en vrouwen  2016 3e kwartaal                    8461   
3   Totaal mannen en vrouwen  2016 4e kwartaal                    8478   
4                     Mannen  2016 1e kwartaal                    4464   
5                     Mannen  2016 2e kwartaal                    4502   
6                     Mannen  2016 3e kwartaal                    4541   
7                     Mannen  2016 4e kwartaal                    4553   
8                    Vrouwen  2016 1e kwartaal                    3823   
9                    Vrouwen  2016 2e kwartaal                    3884   
10                   Vrouwen  2016 3e kwartaal                    3920   
11                   Vrouwen  2016 4e kwartaal                    3925   

    Werknemer  Werknemer met vaste arbeidsrelatie  \textbackslash{}
0        6894                                5146   
1        7000                                5163   
2        7058                                5169   
3        7047                                5155   
4        3584                                2733   
5        3615                                2729   
6        3646                                2713   
7        3655                                2731   
8        3310                                2413   
9        3385                                2435   
10       3412                                2456   
11       3392                                2424   

    Werknemer met flexibele arbeidsrelatie  \textbackslash{}
0                                     1748   
1                                     1836   
2                                     1889   
3                                     1892   
4                                      851   
5                                      886   
6                                      933   
7                                      924   
8                                      897   
9                                      950   
10                                     956   
11                                     968   

   Percentage werknemer met  vaste arbeidsrelatie  \textbackslash{}
0                                          74.64\%   
1                                          73.76\%   
2                                          73.24\%   
3                                          73.15\%   
4                                          76.26\%   
5                                          75.49\%   
6                                          74.41\%   
7                                          74.72\%   
8                                          72.90\%   
9                                          71.94\%   
10                                         71.98\%   
11                                         71.46\%   

   Percentage werknemer met flexibele arbeidsrelatie  
0                                             33.97\%  
1                                             35.56\%  
2                                             36.54\%  
3                                             36.70\%  
4                                             31.14\%  
5                                             32.47\%  
6                                             34.39\%  
7                                             33.83\%  
8                                             37.17\%  
9                                             39.01\%  
10                                            38.93\%  
11                                            39.93\%  

    \end{Verbatim}

    \begin{Verbatim}[commandchars=\\\{\}]
{\color{incolor}In [{\color{incolor}79}]:} \PY{n}{df1}
\end{Verbatim}

            \begin{Verbatim}[commandchars=\\\{\}]
{\color{outcolor}Out[{\color{outcolor}79}]:}                     Geslacht           Periode  Toaal Werzame Personen  \textbackslash{}
         0   Totaal mannen en vrouwen  2016 1e kwartaal                    8287   
         1   Totaal mannen en vrouwen  2016 2e kwartaal                    8386   
         2   Totaal mannen en vrouwen  2016 3e kwartaal                    8461   
         3   Totaal mannen en vrouwen  2016 4e kwartaal                    8478   
         4                     Mannen  2016 1e kwartaal                    4464   
         5                     Mannen  2016 2e kwartaal                    4502   
         6                     Mannen  2016 3e kwartaal                    4541   
         7                     Mannen  2016 4e kwartaal                    4553   
         8                    Vrouwen  2016 1e kwartaal                    3823   
         9                    Vrouwen  2016 2e kwartaal                    3884   
         10                   Vrouwen  2016 3e kwartaal                    3920   
         11                   Vrouwen  2016 4e kwartaal                    3925   
         
             Werknemer  Werknemer met vaste arbeidsrelatie  \textbackslash{}
         0        6894                                5146   
         1        7000                                5163   
         2        7058                                5169   
         3        7047                                5155   
         4        3584                                2733   
         5        3615                                2729   
         6        3646                                2713   
         7        3655                                2731   
         8        3310                                2413   
         9        3385                                2435   
         10       3412                                2456   
         11       3392                                2424   
         
             Werknemer met flexibele arbeidsrelatie  \textbackslash{}
         0                                     1748   
         1                                     1836   
         2                                     1889   
         3                                     1892   
         4                                      851   
         5                                      886   
         6                                      933   
         7                                      924   
         8                                      897   
         9                                      950   
         10                                     956   
         11                                     968   
         
            Percentage werknemer met  vaste arbeidsrelatie  \textbackslash{}
         0                                          74.64\%   
         1                                          73.76\%   
         2                                          73.24\%   
         3                                          73.15\%   
         4                                          76.26\%   
         5                                          75.49\%   
         6                                          74.41\%   
         7                                          74.72\%   
         8                                          72.90\%   
         9                                          71.94\%   
         10                                         71.98\%   
         11                                         71.46\%   
         
            Percentage werknemer met flexibele arbeidsrelatie  
         0                                             33.97\%  
         1                                             35.56\%  
         2                                             36.54\%  
         3                                             36.70\%  
         4                                             31.14\%  
         5                                             32.47\%  
         6                                             34.39\%  
         7                                             33.83\%  
         8                                             37.17\%  
         9                                             39.01\%  
         10                                            38.93\%  
         11                                            39.93\%  
\end{Verbatim}
        
    \begin{Verbatim}[commandchars=\\\{\}]
{\color{incolor}In [{\color{incolor}80}]:} \PY{k+kn}{import} \PY{n+nn}{numpy} \PY{k}{as} \PY{n+nn}{np}
         \PY{k+kn}{import} \PY{n+nn}{matplotlib}\PY{n+nn}{.}\PY{n+nn}{pyplot} \PY{k}{as} \PY{n+nn}{plt}
\end{Verbatim}

    \begin{Verbatim}[commandchars=\\\{\}]
{\color{incolor}In [{\color{incolor}81}]:} \PY{o}{\PYZpc{}}\PY{k}{matplotlib} inline
\end{Verbatim}

    \begin{Verbatim}[commandchars=\\\{\}]
{\color{incolor}In [{\color{incolor}82}]:} \PY{c+c1}{\PYZsh{} Creating 12}
\end{Verbatim}

    \begin{Verbatim}[commandchars=\\\{\}]
{\color{incolor}In [{\color{incolor}83}]:} \PY{k+kn}{import} \PY{n+nn}{pandas} \PY{k}{as} \PY{n+nn}{pd}
         \PY{n}{df1}\PY{o}{=}\PY{n}{pd}\PY{o}{.}\PY{n}{read\PYZus{}csv}\PY{p}{(}\PY{l+s+s1}{\PYZsq{}}\PY{l+s+s1}{datasets/datasets/12.csv}\PY{l+s+s1}{\PYZsq{}}\PY{p}{)}
         \PY{n+nb}{print}\PY{p}{(}\PY{n}{df1}\PY{p}{)}
\end{Verbatim}

    \begin{Verbatim}[commandchars=\\\{\}]
             Periode  Toaal Werzame Personen  Werknemer  \textbackslash{}
0   2016 1e kwartaal                    8287       6894   
1   2016 2e kwartaal                    8386       7000   
2   2016 3e kwartaal                    8461       7058   
3  T2016 4e kwartaal                    8478       7047   

   Werknemer met vaste arbeidsrelatie  Werknemer met flexibele arbeidsrelatie  \textbackslash{}
0                                5146                                    1748   
1                                5163                                    1836   
2                                5169                                    1889   
3                                5155                                    1892   

  Percentage werknemer met  vaste arbeidsrelatie  \textbackslash{}
0                                         74.64\%   
1                                         73.76\%   
2                                         73.24\%   
3                                         73.15\%   

  Percentage werknemer met flexibele arbeidsrelatie  
0                                            33.97\%  
1                                            35.56\%  
2                                            36.54\%  
3                                            36.70\%  

    \end{Verbatim}

    \begin{Verbatim}[commandchars=\\\{\}]
{\color{incolor}In [{\color{incolor}84}]:} \PY{n}{df1}
\end{Verbatim}

            \begin{Verbatim}[commandchars=\\\{\}]
{\color{outcolor}Out[{\color{outcolor}84}]:}              Periode  Toaal Werzame Personen  Werknemer  \textbackslash{}
         0   2016 1e kwartaal                    8287       6894   
         1   2016 2e kwartaal                    8386       7000   
         2   2016 3e kwartaal                    8461       7058   
         3  T2016 4e kwartaal                    8478       7047   
         
            Werknemer met vaste arbeidsrelatie  Werknemer met flexibele arbeidsrelatie  \textbackslash{}
         0                                5146                                    1748   
         1                                5163                                    1836   
         2                                5169                                    1889   
         3                                5155                                    1892   
         
           Percentage werknemer met  vaste arbeidsrelatie  \textbackslash{}
         0                                         74.64\%   
         1                                         73.76\%   
         2                                         73.24\%   
         3                                         73.15\%   
         
           Percentage werknemer met flexibele arbeidsrelatie  
         0                                            33.97\%  
         1                                            35.56\%  
         2                                            36.54\%  
         3                                            36.70\%  
\end{Verbatim}
        
    \begin{Verbatim}[commandchars=\\\{\}]
{\color{incolor}In [{\color{incolor}88}]:} \PY{c+c1}{\PYZsh{} Created the most relevant and best file to proceed with}
\end{Verbatim}

    \begin{Verbatim}[commandchars=\\\{\}]
{\color{incolor}In [{\color{incolor}89}]:} \PY{c+c1}{\PYZsh{}import libraries}
         \PY{k+kn}{import} \PY{n+nn}{pandas} \PY{k}{as} \PY{n+nn}{pd}
         \PY{k+kn}{import} \PY{n+nn}{matplotlib}\PY{n+nn}{.}\PY{n+nn}{pyplot} \PY{k}{as} \PY{n+nn}{plt}
         \PY{k+kn}{import} \PY{n+nn}{matplotlib}\PY{n+nn}{.}\PY{n+nn}{dates} \PY{k}{as} \PY{n+nn}{mdates}
         \PY{o}{\PYZpc{}}\PY{k}{matplotlib} inline
         
         \PY{c+c1}{\PYZsh{}picking my variables}
         
         \PY{n}{data} \PY{o}{=} \PY{n}{pd}\PY{o}{.}\PY{n}{read\PYZus{}csv}\PY{p}{(}\PY{l+s+s1}{\PYZsq{}}\PY{l+s+s1}{datasets/datasets/12.csv}\PY{l+s+s1}{\PYZsq{}}\PY{p}{,} \PY{n}{usecols}\PY{o}{=}\PY{p}{[}\PY{l+s+s1}{\PYZsq{}}\PY{l+s+s1}{Periode}\PY{l+s+s1}{\PYZsq{}}\PY{p}{,}\PY{l+s+s1}{\PYZsq{}}\PY{l+s+s1}{Toaal Werzame Personen}\PY{l+s+s1}{\PYZsq{}}\PY{p}{,}\PY{l+s+s1}{\PYZsq{}}\PY{l+s+s1}{Werknemer}\PY{l+s+s1}{\PYZsq{}}\PY{p}{,}\PY{l+s+s1}{\PYZsq{}}\PY{l+s+s1}{Werknemer met vaste arbeidsrelatie}\PY{l+s+s1}{\PYZsq{}}\PY{p}{,}\PY{l+s+s1}{\PYZsq{}}\PY{l+s+s1}{Werknemer met flexibele arbeidsrelatie}\PY{l+s+s1}{\PYZsq{}}\PY{p}{]}\PY{p}{,} \PY{n}{parse\PYZus{}dates}\PY{o}{=}\PY{p}{[}\PY{l+s+s1}{\PYZsq{}}\PY{l+s+s1}{Periode}\PY{l+s+s1}{\PYZsq{}}\PY{p}{]}\PY{p}{)}
         
         \PY{n}{data}\PY{o}{.}\PY{n}{set\PYZus{}index}\PY{p}{(}\PY{l+s+s1}{\PYZsq{}}\PY{l+s+s1}{Periode}\PY{l+s+s1}{\PYZsq{}}\PY{p}{,}\PY{n}{inplace}\PY{o}{=}\PY{k+kc}{True}\PY{p}{)}
         
         \PY{c+c1}{\PYZsh{}plot my data}
         \PY{n}{fig}\PY{p}{,} \PY{n}{ax} \PY{o}{=} \PY{n}{plt}\PY{o}{.}\PY{n}{subplots}\PY{p}{(}\PY{n}{figsize}\PY{o}{=}\PY{p}{(}\PY{l+m+mi}{12}\PY{p}{,}\PY{l+m+mi}{18}\PY{p}{)}\PY{p}{)}
         \PY{n}{data}\PY{o}{.}\PY{n}{plot}\PY{p}{(}\PY{n}{ax}\PY{o}{=}\PY{n}{ax}\PY{p}{)}
\end{Verbatim}

            \begin{Verbatim}[commandchars=\\\{\}]
{\color{outcolor}Out[{\color{outcolor}89}]:} <matplotlib.axes.\_subplots.AxesSubplot at 0x23d168b1c18>
\end{Verbatim}
        
    \begin{center}
    \adjustimage{max size={0.9\linewidth}{0.9\paperheight}}{output_42_1.png}
    \end{center}
    { \hspace*{\fill} \\}
    
    \begin{Verbatim}[commandchars=\\\{\}]
{\color{incolor}In [{\color{incolor}94}]:} \PY{n}{data}\PY{o}{.}\PY{n}{plot}\PY{p}{(}\PY{n}{kind}\PY{o}{=}\PY{l+s+s1}{\PYZsq{}}\PY{l+s+s1}{bar}\PY{l+s+s1}{\PYZsq{}}\PY{p}{,} \PY{n}{ax}\PY{o}{=}\PY{n}{ax}\PY{p}{)}
\end{Verbatim}

            \begin{Verbatim}[commandchars=\\\{\}]
{\color{outcolor}Out[{\color{outcolor}94}]:} <matplotlib.axes.\_subplots.AxesSubplot at 0x23d168b1c18>
\end{Verbatim}
        
    \begin{Verbatim}[commandchars=\\\{\}]
{\color{incolor}In [{\color{incolor}97}]:} \PY{c+c1}{\PYZsh{}import libraries}
         \PY{k+kn}{import} \PY{n+nn}{pandas} \PY{k}{as} \PY{n+nn}{pd}
         \PY{k+kn}{import} \PY{n+nn}{matplotlib}\PY{n+nn}{.}\PY{n+nn}{pyplot} \PY{k}{as} \PY{n+nn}{plt}
         \PY{k+kn}{import} \PY{n+nn}{matplotlib}\PY{n+nn}{.}\PY{n+nn}{dates} \PY{k}{as} \PY{n+nn}{mdates}
         \PY{o}{\PYZpc{}}\PY{k}{matplotlib} inline
         
         \PY{c+c1}{\PYZsh{}picking my variables}
         
         \PY{n}{data} \PY{o}{=} \PY{n}{pd}\PY{o}{.}\PY{n}{read\PYZus{}csv}\PY{p}{(}\PY{l+s+s1}{\PYZsq{}}\PY{l+s+s1}{datasets/datasets/12.csv}\PY{l+s+s1}{\PYZsq{}}\PY{p}{,} \PY{n}{usecols}\PY{o}{=}\PY{p}{[}\PY{l+s+s1}{\PYZsq{}}\PY{l+s+s1}{Periode}\PY{l+s+s1}{\PYZsq{}}\PY{p}{,}\PY{l+s+s1}{\PYZsq{}}\PY{l+s+s1}{Toaal Werzame Personen}\PY{l+s+s1}{\PYZsq{}}\PY{p}{,}\PY{l+s+s1}{\PYZsq{}}\PY{l+s+s1}{Werknemer}\PY{l+s+s1}{\PYZsq{}}\PY{p}{,}\PY{l+s+s1}{\PYZsq{}}\PY{l+s+s1}{Werknemer met vaste arbeidsrelatie}\PY{l+s+s1}{\PYZsq{}}\PY{p}{,}\PY{l+s+s1}{\PYZsq{}}\PY{l+s+s1}{Werknemer met flexibele arbeidsrelatie}\PY{l+s+s1}{\PYZsq{}}\PY{p}{]}\PY{p}{,} \PY{n}{parse\PYZus{}dates}\PY{o}{=}\PY{p}{[}\PY{l+s+s1}{\PYZsq{}}\PY{l+s+s1}{Periode}\PY{l+s+s1}{\PYZsq{}}\PY{p}{]}\PY{p}{)}
         
         \PY{n}{data}\PY{o}{.}\PY{n}{set\PYZus{}index}\PY{p}{(}\PY{l+s+s1}{\PYZsq{}}\PY{l+s+s1}{Periode}\PY{l+s+s1}{\PYZsq{}}\PY{p}{,}\PY{n}{inplace}\PY{o}{=}\PY{k+kc}{True}\PY{p}{)}
         
         \PY{c+c1}{\PYZsh{}plot my data}
         \PY{n}{fig}\PY{p}{,} \PY{n}{ax} \PY{o}{=} \PY{n}{plt}\PY{o}{.}\PY{n}{subplots}\PY{p}{(}\PY{n}{figsize}\PY{o}{=}\PY{p}{(}\PY{l+m+mi}{12}\PY{p}{,}\PY{l+m+mi}{18}\PY{p}{)}\PY{p}{)}
         \PY{n}{data}\PY{o}{.}\PY{n}{plot}\PY{p}{(}\PY{n}{kind}\PY{o}{=}\PY{l+s+s1}{\PYZsq{}}\PY{l+s+s1}{bar}\PY{l+s+s1}{\PYZsq{}}\PY{p}{,} \PY{n}{ax}\PY{o}{=}\PY{n}{ax}\PY{p}{)}
\end{Verbatim}

            \begin{Verbatim}[commandchars=\\\{\}]
{\color{outcolor}Out[{\color{outcolor}97}]:} <matplotlib.axes.\_subplots.AxesSubplot at 0x23d168e44e0>
\end{Verbatim}
        
    \begin{center}
    \adjustimage{max size={0.9\linewidth}{0.9\paperheight}}{output_44_1.png}
    \end{center}
    { \hspace*{\fill} \\}
    
    \begin{Verbatim}[commandchars=\\\{\}]
{\color{incolor}In [{\color{incolor} }]:} 
\end{Verbatim}


    % Add a bibliography block to the postdoc
    
    
    
    \end{document}
